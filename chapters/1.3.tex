\section{Mathemathik}

Das Einbinden von Formeln, Gleichungen oder auch Brüchen wird durch LaTeX und bestimmten Paketen, die importiert werden übernommen.
Diese ermöglichen es, dass Formeln in dem aktuellen Paragraphen eingebunden werden können \(a^2 + b^2 = c \).
Das Absetzen der mathematischen Ausdrücke ist durch eckige Klammern möglich. Dies soll hier anhand einer grafisch komplexeren Gleichung demonstriert werden.
Die Darstellung wird von LaTeX entsprechend angepasst.

\[
 \frac{n!}{k!(n-k)!} = \binom{n}{k}
\]

\noindent Auch das Einfügen von Matrizen wird von LaTex erleichtert.
Als Beispiel soll folgende Matrix dienen.

\[
 A_{m,n} =
 \begin{pmatrix}
    a_{1,1} & a_{1,2} & \cdots & a_{1,n} \\
    a_{2,1} & a_{2,2} & \cdots & a_{2,n} \\
    \vdots  & \vdots  & \ddots & \vdots  \\
    a_{m,1} & a_{m,2} & \cdots & a_{m,n}
 \end{pmatrix}
\]

\noindent Weitere Möglichkeiten sind auf der Seite \url{http://en.wikibooks.org/wiki/LaTeX/Mathematics} zu finden.